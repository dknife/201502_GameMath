\documentclass{beamer}
%
% Choose how your presentation looks.
%
% For more themes, color themes and font themes, see:
% http://deic.uab.es/~iblanes/beamer_gallery/index_by_theme.html
%
\mode<presentation>
{
  \usetheme{Madrid}      % or try Darmstadt, Madrid, Warsaw, ...
  \usecolortheme{seahorse} % or try albatross, beaver, crane, ...
  \usefonttheme{serif}  % or try serif, structurebold, ...
  \setbeamertemplate{navigation symbols}{}
  \setbeamertemplate{caption}[numbered]
} 


\usepackage[english]{babel}
\usepackage{kotex}
%\usepackage[utf8x]{inputenc}

\title[게임수학 - 벡터연산]{ 게임 수학 강의 노트 03 - 벡터 연산}
\author{강영민}
\institute{동명대학교}
\date{2015년 2학기}

\begin{document}

%%%%%%%%%%%%%%%%%%%%%%%%%%%%%%%%%%%%%%%%%%%%%%%%%%%%%%%%%
\begin{frame}
  \titlepage
\end{frame}

% Uncomment these lines for an automatically generated outline.
%\begin{frame}{Outline}
%  \tableofcontents
%\end{frame}


%%%%%%%%%%%%%%%%%%%%%%%%%%%%%%%%%%%%%%%%%%%%%%%%%%%%%%%%%
\begin{frame}{벡터의 덧셈}
$\mathbf v + \mathbf w = (v_x + w_x, v_y + w_y , v_z + w_z )$
\begin{figure}
\includegraphics[width=8cm]{Math_vector/vectorAdd.eps}
\end{figure}
\end{frame}
%%%%%%%%%%%%%%%%%%%%%%%%%%%%%%%%%%%%%%%%%%%%%%%%%%%%%%%%%

%%%%%%%%%%%%%%%%%%%%%%%%%%%%%%%%%%%%%%%%%%%%%%%%%%%%%%%%%
\begin{frame}{벡터의 뺄셈}
$\mathbf v - \mathbf w = (v_x - w_x, v_y - w_y , v_z - w_z )$
\\

\begin{figure}
\includegraphics[width=12cm]{Math_vector/vectorSub.eps}
\end{figure}
\end{frame}
%%%%%%%%%%%%%%%%%%%%%%%%%%%%%%%%%%%%%%%%%%%%%%%%%%%%%%%%%

%%%%%%%%%%%%%%%%%%%%%%%%%%%%%%%%%%%%%%%%%%%%%%%%%%%%%%%%%
\begin{frame}{벡터에 스칼라 곱하기}

벡터는 크기만을 가진 스칼라와 곱할 수 있다. 어떤 스칼라 값 $s$가 있다고 하자, 이 스칼라 값과 벡터 $\mathbf v = (v_x , v_y, v_z)$를 곱한 $s \mathbf v$는 다음과 같다.
$$ s \mathbf v = (s v_x , s v_y , s v_z )$$
\end{frame}
%%%%%%%%%%%%%%%%%%%%%%%%%%%%%%%%%%%%%%%%%%%%%%%%%%%%%%%%%

%%%%%%%%%%%%%%%%%%%%%%%%%%%%%%%%%%%%%%%%%%%%%%%%%%%%%%%%%
\begin{frame}{벡터의 기본적인 연산 규칙}

\begin{eqnarray}
\mathbf a + \mathbf b = \mathbf b + \mathbf a \nonumber \\
(\mathbf a + \mathbf b) + \mathbf c = \mathbf a + (\mathbf b + \mathbf c) \nonumber \\
\mathbf a + \vec{0} = \vec{0} + \mathbf a = \mathbf a \nonumber \\
\mathbf a + (- \mathbf a) = \mathbf a - \mathbf a = \vec{0} \nonumber \\
(k+l) \mathbf a = k \mathbf a + l \mathbf a \nonumber \\
(kl) \mathbf a = k (l \mathbf a) \nonumber \\
1 \mathbf a = \mathbf a \nonumber \\
0 \mathbf a = \vec{0} \nonumber \\
(-1) \mathbf a = - \mathbf a \nonumber
\end{eqnarray}
\end{frame}
%%%%%%%%%%%%%%%%%%%%%%%%%%%%%%%%%%%%%%%%%%%%%%%%%%%%%%%%%

%%%%%%%%%%%%%%%%%%%%%%%%%%%%%%%%%%%%%%%%%%%%%%%%%%%%%%%%%
\begin{frame}{벡터의 스칼라 곱, 혹은 내적(dot product)}
\begin{itemize}
\item 내적
	\begin{itemize}
	\item 스칼라 곱(scalar product)라고도 부름 
	\item 두 개의 벡터를 피연산자(operand)로 하는 이항 연산(binary operator)로서 그 결과가 스칼라 값
	\item 두 벡터 $\mathbf a$와 $\mathbf b$의 내적은 $\mathbf a \cdot \mathbf b$로 표현
	\item 두 벡터가 이루는 사잇각이 $\theta$라고 하며, 내적의 크기는 다음과 같다.
		\begin{itemize}
		\item $\mathbf a \cdot \mathbf b = || \mathbf a || ||\mathbf b || \cos \theta$
		\end{itemize}
	\item 실제 계산 방법
		\begin{itemize}
		\item $\mathbf a , \mathbf b \in \mathbb R^n $
		\item $\mathbf a \cdot \mathbf b = a_1 b_1 + a_2 b_2 + \cdots + a_n b_n = \sum_{i=1}^n a_i b_i $
		\end{itemize}
	\end{itemize}
\end{itemize}

\end{frame}
%%%%%%%%%%%%%%%%%%%%%%%%%%%%%%%%%%%%%%%%%%%%%%%%%%%%%%%%%

%%%%%%%%%%%%%%%%%%%%%%%%%%%%%%%%%%%%%%%%%%%%%%%%%%%%%%%%%
\begin{frame}{벡터 내적의 의미}
\begin{figure}
\includegraphics[width=12cm]{Math_vector/innerProduct.eps}
\end{figure}

\end{frame}
%%%%%%%%%%%%%%%%%%%%%%%%%%%%%%%%%%%%%%%%%%%%%%%%%%%%%%%%%


%%%%%%%%%%%%%%%%%%%%%%%%%%%%%%%%%%%%%%%%%%%%%%%%%%%%%%%%%
\begin{frame}{벡터 내적의 활용}
\begin{itemize}
\item 코사인 함수의 특성을 통해 간단히 얻어지는 사실
	\begin{itemize}
	\item $\theta = 0 \Rightarrow \cos \theta = 1 , \mathbf a \cdot \mathbf b = ||\mathbf a|| ||\mathbf b|| $
	\item $\theta = \pi / 2 \Rightarrow \cos \theta = 0, \mathbf a \cdot \mathbf b = 0 $
	\item $\mathbf a \cdot \mathbf a =  ||\mathbf a||^2$
	\end{itemize}
\end{itemize}


\begin{itemize}
\item 벡터를 이용하여 각도를 계산하거나 투영을 계산하는 데에 널리 사용
	\begin{itemize}
	\item $ \mathbf v \cdot \mathbf w  =  ||\mathbf v|| ||\mathbf w|| \cos \theta $
	\item $ \cos \theta = \frac{\mathbf v \cdot \mathbf w}{||\mathbf v||||\mathbf w||} $
	\item $ \theta  = \cos^{-1} \frac{\mathbf v \cdot \mathbf w}{||\mathbf v||||\mathbf w||} $
	\item $ \theta =  \cos^{-1}\frac{ v_x w_x + v_y w_y + v_z w_y}{\sqrt{v_x^2 + v_y^2 + v_z^2} \sqrt{w_x^2 + w_y^2 + w_z^2} }$
	\end{itemize} 
\end{itemize}

\end{frame}
%%%%%%%%%%%%%%%%%%%%%%%%%%%%%%%%%%%%%%%%%%%%%%%%%%%%%%%%%


%%%%%%%%%%%%%%%%%%%%%%%%%%%%%%%%%%%%%%%%%%%%%%%%%%%%%%%%%
\begin{frame}{벡터 내적의 활용}
\hrule

\noindent \colorbox{lightgray}{\begin{minipage}{6cm}예제\end{minipage}} 


\noindent 어떤 두 벡터가 각각 (3,2)와 (4,1)이라고 하자. 두 벡터가 이루는 각도를 구하라.

\noindent \colorbox{lightgray}{\begin{minipage}{6cm}정답\end{minipage}} 

\noindent 두 벡터를 각각 $\mathbf v$와 $\mathbf w$로 표현하자. 두 벡터의 내적 $\mathbf v \cdot \mathbf w$는 $3 \cdot 4 + 2 \cdot 1$, 즉 14이다.
각각의 길이는 $||\mathbf v|| = \sqrt{9+4} = \sqrt{13}$과 $||\mathbf w|| = \sqrt{16+1} = \sqrt{17}$이다.
따라서 두 벡터의 사이각은 다음과 같다.
$$\theta = \cos^{-1} \frac{14}{\sqrt{13} \sqrt{17}} = \cos^{-1} \frac{14}{\sqrt{221}} \simeq \cos^{-1} 0.94174191159484 \simeq 19.65 ^{\circ}$$

\hrule

\end{frame}
%%%%%%%%%%%%%%%%%%%%%%%%%%%%%%%%%%%%%%%%%%%%%%%%%%%%%%%%%

\end{document}





\subsection{내적을 이용한 벡터의 투영}

어떤 벡터 $\mathbf v$가 $(k, l)$로 표현된다는 것은 $xy$ 직교 좌표계에서 기저벡터(basis vector)가 되는
$x$축 방향의 단위벡터와 $y$축 방향의 단위벡터의 크기를 적절히 변경하여
더한 것과 같다. 이를 그림 \ref{fig:vector:vectorComponents}에서 살펴보자.
$x$축 단위벡터를 $\mathbf u$라고 하고, $y$축 단위벡터를 $\mathbf v$라고 하면, 벡터 $\mathbf v$는 $k \mathbf u$와 $l \mathbf v$를 더하여
얻을 수 있다.


\begin{figure}[h!]
  \centering
    \includegraphics[width=10cm]{Math_vector/vectorComponents.eps}
    \caption{기저벡터의 크기변경 양으로서의 벡터 성분}
    \label{fig:vector:vectorComponents}
\end{figure}

이제 축을 변경하여 그림 \ref{fig:vector:vectorComponentsArb}와 같이 새로운 직교 좌표계를 고려해 보자.
여기서는 두 축이 $\mathbf u$와 $\mathbf v$이다. 이 두 축으로 표현하기 위해서는
원래의 벡터 $\mathbf a$를 $\mathbf u$ 축으로 투영했을 때의 길이 $\alpha$와 $\mathbf v$ 축으로 투영했을 때의 길이 $\beta$를 알아야 한다.
이 값을 알게 되면 $\mathbf a$는 두 축을 나타내는 기저 벡터들의 합성으로 다음과 같이 표현된다.

$$\mathbf a = \alpha \mathbf u + \beta \mathbf v$$

이것은 이 두 축을 기준으로 하는 좌표계에서는 $\mathbf a$가 $(\alpha, \beta)$의 좌표로 표현된다는 것이다.


\begin{figure}[h!]
  \centering
    \includegraphics[width=10cm]{Math_vector/vectorComponentsArb.eps}
    \caption{변경된 직교 좌표계의 기저 벡터의 합성으로 표현된 벡터}
    \label{fig:vector:vectorComponentsArb}
\end{figure}

그림 \ref{fig:vector:vectorComponentsArb}의 $\mathbf a$가 축 $\mathbf u$와 $\mathbf v$ 방향으로 가지는 길이 $\alpha$와 $\beta$는 어떻게 구할 수 있을까? 이때 내적을 이용할 수 있다. $\alpha$는 $\mathbf a$를 $\mathbf u$ 방향으로 투영한 그림자의 길이이므로,
$\mathbf a \cdot \mathbf u / || \mathbf u ||$가 된다. 축은 단위 벡터로 표현하므로 $||\mathbf u||=1$이므로,
$\alpha = \mathbf a \cdot \mathbf u$이다.

비슷한 방법으로 $\beta = \mathbf a \cdot \mathbf v$임을 알 수 있다. 따라서 $\mathbf u$와 $\mathbf v$를 축으로 하는 좌표계에서 $\mathbf a$는
$(\mathbf a \cdot \mathbf u, \mathbf a \cdot \mathbf v)$의 좌표로 표현된다고 할 수 있다.


\hrule

\noindent \colorbox{lightgray}{\begin{minipage}{6cm}예제\end{minipage}} 


\noindent 그림 \ref{fig:vector:vecProjection}처럼 어떤 벡터 $\mathbf a$가 (4,5)이고, 다른 벡터 $\mathbf b$는 (10,2)라고 하자.
이때 벡터 $\mathbf a$를 $\mathbf b$ 위에 수직방향으로 내린 그림자가 되는 벡터 $\mathbf a_{prj}$을 구하라.

\begin{figure}[h!]
  \centering
    \includegraphics[width=8cm]{Math_vector/vecProjection.eps}
    \caption{다른 벡터 위로 투영된 벡터 구하기}
    \label{fig:vector:vecProjection}
\end{figure}

\noindent \colorbox{lightgray}{\begin{minipage}{6cm}정답\end{minipage}} 

\noindent $\mathbf a_{prj}$은 그 방향이 $\mathbf b$와 같고, 길이는 투영된 그림자의 길이 $l$이다.
방향은 단위 벡터로 구할 수 있으므로 $\mathbf b$를 정규화한 $\mathbf b/||\mathbf b||$가 방향벡터이다. 이를 간단히 $\tilde{\mathbf b}$로 쓰자.
그러면 $\mathbf a_{prj}$은 $l \tilde{\mathbf b}$가 된다. $l$은 내적을 이용하여 구할 수 있다. 그 값은 다음과 같다.
$$l = \mathbf a \cdot \mathbf b / || \mathbf b||$$

그림에 표시된 바와 같이 $\mathbf a$와 $\mathbf b$의 길이는 각각 $\sqrt{41}$과 $\sqrt{104}$이며, 두 벡터의 내적은 50이므로 다음과 같이 
$\mathbf a_{prj}$를 구할 수 있다.

\begin{eqnarray}
\mathbf a_{prj} & = & l \tilde{\mathbf b} \\ \nonumber
& = & 50 / \sqrt{104} \tilde{\mathbf b} \\ \nonumber
& = & \frac{50}{\sqrt{104}} \frac{(10,2)}{\sqrt{104}}\\ \nonumber
& = & \frac{50}{104} (10,2) \\ \nonumber
& \simeq & 0.48 (10,2) \\ \nonumber
& \simeq & (4.8, 0.96) \\ \nonumber
\end{eqnarray}

\hrule

\section{벡터의 외적(外積)}

벡터의 외적(cross product)은 `벡터 곱(vector product)'라고도 한다. 이 연산은 두 벡터를 피연산자로 하는 이항연산으로 그 결과가 벡터가 되는 곱이다 \cite{wiki:vectorProduct}. 벡터를 곱해 행렬을 얻는 외적(outer product)과 용어의 혼동이 있다. 이와 관련해서는 참고문헌 \cite{wiki:outerProduct}를 참고하라.


어떤 두 벡터 $\mathbf a$와 $\mathbf b$의 외적은 $\mathbf a \times \mathbf b$로 표현한다. 
그 결과는 벡터이기 때문에 $k \mathbf n$이 된다. 여기서 $\mathbf n$은 $\mathbf a$와 $\mathbf b$에 동시에 수직인 단위벡터이다. 
동시에 수직인 벡터는 두 가지가 있는데, 그 방향이 서로 반대이다. 이 방향의 결정은 좌표계가 왼손 좌표계인가 오른손 좌표계인가에 따라 달라진다. 
오른손 좌표계에서 얻어지는 방향을 $\mathbf n_r$, 왼손 좌표계에서 얻어지는 방향을 $\mathbf n_l$이라고 하면 그림 \ref{fig:vector:crossProductDir}와
같이 방향이 정해진다.

\begin{figure}[h!]
  \centering
    \includegraphics[width=6cm]{Math_vector/crossProductDir.eps}
    \caption{벡터 외적의 방향}
    \label{fig:vector:crossProductDir}
\end{figure}

결과로 얻어진 벡터의 크기 $k$는 두 벡터의 곱이므로 두 벡터의 크기에 비례하며, 내적이 두 벡터 사잇각의 코사인에 비례했던 것에 반해 외적의 크기는
두 벡터 사잇각의 사인(sine) 값에 비례한다. 그 크기는 다음과 같다.

$$k = ||\mathbf a|| |\mathbf b|| \sin \theta$$

따라서 두 벡터의 외적은 다음과 같이 표현할 수 있다.

\begin{eqnarray}
\mathbf a \times \mathbf b = ||\mathbf a|| |\mathbf b|| \sin \theta \mathbf n
\label{eq:crossProductComputation}
\end{eqnarray}

이 식이 가진 의미를 다시 살펴보면 외적은 두 벡터에 동시에 수직한 벡터를 구할 수 있으며, 이렇게 얻어진 벡터의 크기는 두 벡터의 크기에 비례하고,
두 벡터가 수직일 때에 최대치, 서로 같은 방향이나 반대방향일 때에 최소치가 됨을 알 수 있다. 이를 그림 \ref{fig:vector:crossProductDirSize}으로 살펴보자.
그림에는 벡터  $\mathbf a$와 $\mathbf b$가 서로 다른 각도를 이루고 있을 때에 외적이 어떤 방향과 크기를 갖는지를 보이고 있다.
왼쪽의 세 가지 예는 두 벡터가 예각을 이루고 있다. 오른손 좌표계라고 하면 이 경우 위로 향하는 수직 벡터가 외적의 방향이 된다.
마지막 예에서는 두 벡터가 이루는 각이 둔각이 되는데, 이 경우 외적의 방향은 예각일 때와 반대로 아래를 향하고 있다. 
외적의 크기는 벡터 $\mathbf a$와 $\mathbf b$의 끝점을 연결하고 두 벡터를 선분으로 하는 삼각형의 넓이 $S$에 비례한다. 정확하게는 다음과 같은 관계를 가진다.

$$|| \mathbf a \times \mathbf b || = \frac{1}{2} S $$

\begin{figure}[h!]
  \centering
    \includegraphics[width=12cm]{Math_vector/crossProductDirSize.eps}
    \caption{벡터 $\mathbf a$와 $\mathbf b$의 외적이 갖는 방향과 크기}
    \label{fig:vector:crossProductDirSize}
\end{figure}


\subsection{외적의 계산}

식 \ref{eq:crossProductComputation}에 나타난 외적 표현은 외적이 가진 기하적 의미를 파악하는 데에 도움이 된다. 하지만 외적을 실제로 계산하기 위해 사잇각을 구하고, 그 사인(sine) 값을 구하는 방식으로
계산을 하지는 않는다. 3차원 벡터의 외적을 구하는 문제부터 살펴 보자.

3차원 벡터 $\mathbf a$와 $\mathbf b$가 다음과 같은 성분을 갖는다고 가정하자.

$$\mathbf a=(a_1, a_2, a_3), \mathbf b=(b_1, b_2, b_3)$$


이때, 두 벡터의 외적은 다음과 같이 구해진다.

$$\mathbf a \times \mathbf b = (a_2 b_3 - a_3 b_2, a_3 b_1 - a_1 b_3, a_1 b_2 - a_2 b_1)$$

이것은 다음 장에서 다룰 ``행렬"을 이용한 곱셈으로 표현이 가능하다. 행렬을 다음 장에서 다룰 예정이기 때문에
행렬과 관련된 용어를 최소로 하여 설명해 보자.

두 벡터 $\mathbf a$와 $\mathbf b$의 외적은
$\mathbf a$의 성분을 이용하여 ``반대칭(skew-symmetric 혹은 antisymmetric)" 행렬을 만들어 구할 수 있다. 반대칭 행렬은
행과 열을 뒤집으면 원래 행렬의 부호를 바꾼 것과 같은 행렬이다. 반대칭 행렬은 이 조건만 만족하면 여러 종류를 만들 수 있는데,
주어진 벡터 $\mathbf a=(a_1, a_2, a_3)$의 각 성분을 이용한 반대칭 행렬을 하나 만들어 보자.
반대칭 행렬의 특성을 유지하기 위해서는 대각 성분이 반드시 0이어야 한다. 대각 성분을 기준으로 대칭의 위치에 있는 성분들은
서로 반대 부호를 가진다. 이런 예가 가능할 것이다. 대각 성분은 0으로 나머지 성분의 자리를 서로 바꾸고 음수 기호를 한 칸씩 이동하여 설정하였다.

\begin{eqnarray}
\mathbf A^* = \left ( 
\begin{array}{ccc}
0 & -a_3 & a_2 \\
a_3 & 0 & -a_1 \\
-a_2 & a_1 & 0
\end{array}
\right )
\label{eq:skewMatrixEx}
\end{eqnarray}

식 \ref{eq:skewMatrixEx}에 나타난 행렬을 $\mathbf b$에 곱해 보자.

\begin{eqnarray}
\mathbf A^* \mathbf b = \left ( 
\begin{array}{ccc}
0 & -a_3 & a_2 \\
a_3 & 0 & -a_1 \\
-a_2 & a_1 & 0
\end{array}
\right ) \mathbf b
= (a_2 b_3 - a_3 b_2, a_3 b_1 - a_1 b_3, a_1 b_2 - a_2 b_1 )
\label{eq:crossProductWSkewMatrix}
\end{eqnarray}

식 \ref{eq:crossProductWSkewMatrix}의 결과는 외적과 일치함을 확인할 수 있다.

이제 4차원 이상의 고차 공간에서 외적을 고려해 보자.
3차원 외적과 같이 두 벡터의 외적이 두 벡터 모두에 직교(orthogonal)하면서, 외적 연산이 반교환법칙(anti-communtative)을 만족하는 경우는
7차원 하나 뿐임이 알려져 있다 \cite{wiki:sevenDimensionCrossProduct}.
이때 사용되는 반대칭 행렬은 다음과 같다.

\begin{eqnarray}
\mathbf A^* = \left ( 
\begin{array}{ccccccc}
0 & a_3 & -a_2 & a_5 & -a_4 & -a_7 & a_6 \\
-a_3 & 0 & a_1 & a_6 & a_7 & -a_4 & -a_5 \\
a_2 & -a_1 & 0 & a_7 & -a_6 & a_5 & -a_4 \\
-a_5 & -a_6 & -a_7 & 0 & a_1 & a_2 & a_3 \\
a_4 & -a_7 & a_6 & -a_1 & 0 & -a_3 & a_2 \\
a_7 & a_4 & -a_5 & -a_2 & a_3 & 0 & -a_1 \\
-a_6 & a_5 & a_4 & -a_3 & -a_2 & a_1 & 0 \\
\end{array}
\right )
\label{eq:skewMatrixEx}
\end{eqnarray}

실제로 외적의 기하적 특성을 활용하여 게임이나 그래픽스 프로그램에서 응용하는 경우는 3차원 공간이 대부분이다.
따라서 우리는 벡터의 외적 문제를 3차원 공간에서의 문제로 한정해서 다룰 것이다.

3차원 공간에서 벡터의 외적은 반교환법칙이 성립한다. 이는 두 벡터의 순서를 바꾸어 외적 연산을 하면 그 결과는 원래 결과의 반대 부호를 갖는다는 것이다.
외적 연산과 관련된 법칙들은 다음과 같다.

\begin{eqnarray}
\mathbf a \times \mathbf b = - \mathbf b \times \mathbf a \\ \nonumber
\mathbf a \times ( \mathbf b \times \mathbf c) = (\mathbf a \times \mathbf b) \times \mathbf c \\ \nonumber
(\mathbf a + \mathbf b ) \times \mathbf c = (\mathbf a \times \mathbf c) + (\mathbf b \times \mathbf c) \\ \nonumber
k(\mathbf a \times \mathbf b) = k \mathbf a \times \mathbf b = \mathbf a \times k \mathbf b \\ \nonumber
\mathbf a \times \vec{0} = \vec{0} \times \mathbf a = \vec{0} \\ \nonumber
\mathbf a \times \mathbf a = \vec{0} \\ \nonumber
\mathbf a \cdot (  \mathbf a \times \mathbf b) = \vec{0} \\ \nonumber
\mathbf b \cdot ( \mathbf a \times \mathbf b) = \vec{0}
\end{eqnarray}

\subsection{2차원 공간에서의 외적과 외적의 응용}

2차원 공간의 두 벡터 $\mathbf a=(a_x, a_y)$와 $\mathbf b=(b_x, b_y)$의 외적을 구할 수 있을까?
외적은 앞서 설명한 바와 같이 두 벡터에 동시에 수직인 벡터를 결과로 생성한다.
그런데 2차원 공간의 어떤 벡터도 두 개의 2차원 벡터에 동시에 수직일 수가 없다.
그림 \ref{fig:vector:crossProduct2D}를 살펴 보자. 그림의 회색 면은 2차원 공간의 일부이다.
이 공간에서 정의되는 두 벡터 $\mathbf a$와 $\mathbf b$의 외적은 그림에서 볼 수 있는 바와 같이
2차원 공간 밖에서 정의된다. 이 외적은 평면에 수직이므로 $x$ 성분과 $y$ 성분이 모두 0이다.
따라서 이 벡터를 정의한다면 새로운 축 $z$가 필요하며, 이 $z$ 축 성분으로만 표현이 된다.
외적의 크기가 $\delta$라면, 하나뿐인 $z$축 성분이 바로 $\delta$가 되어야 한다.


\begin{figure}[h!]
  \centering
    \includegraphics[width=12cm]{Math_vector/crossProduct2D.eps}
    \caption{2차원 벡터 $\mathbf a$와 $\mathbf b$의 외적}
    \label{fig:vector:crossProduct2D}
\end{figure}


2차원 벡터의 내적이 2차원 공간 밖에 정의가 되고, 이것은 새로운 차원을 요구하므로 
2차원 벡터의 내적은 3차원 벡터라고 볼 수 있다.
이것은 2차원 벡터 $\mathbf a = (a_x, a_y)$와 $\mathbf b= (b_x, b_y)$를 3차원 벡터로 가정하여 $z$ 축 성분을 0으로 놓고 다음과 같이 
외적을 구한 결과와 같다. 

$$\mathbf a = (a_x , a_y, 0)$$
$$\mathbf b = (b_x , b_y, 0)$$
$$\mathbf a \times \mathbf b = (0, 0, a_x b_y - a_y b_x )$$

여기서 마지막 $z$ 성분의 값으로 여러 가지를 알 수가 있다. 이 값을 $\delta$라고 하면 $\delta>0$인 경우는 $\mathbf b$가 $\mathbf a$의 진행 방향을 기준으로
왼쪽에 있다는 것을 의미하며, 반대로 $\delta<0$인 경우는 오른쪽에 있다는 것이다.
그리고 그 절대값은 앞서 설명한 바와 같이 두 벡터 사이에 만들어지는 삼각형의 크기에 비례한다.



\hrule

\noindent \colorbox{lightgray}{\begin{minipage}{6cm}예제\end{minipage}} 


\noindent  세 개의 꼭지점 좌표가 (2,1), (8,4), (4,8)이라고 할 때,
이 삼각형의 넓이 $S$를 구하라.

\noindent \colorbox{lightgray}{\begin{minipage}{6cm}정답\end{minipage}} 

그림 \ref{fig:vector:triangleArea}를 보자. 이 문제를 벡터를 이용하여 풀기 위해서는 꼭지점의 좌표가 아니라
이 삼각형을 만들어 내는 두 개의 벡터가 필요하다. 꼭지점들을 각각 $\mathbf a, \mathbf b, \mathbf c$로 표현하자.
우리는 $\mathbf a$에서 $\mathbf b$로 가는 벡터를 구할 수 있고, 이를 $\mathbf u$라고 부를 수 있다. 이 벡터는 (6,3)이 된다.
비슷한 방식으로 $\mathbf a$에서 $\mathbf c$로 가는 벡터를 $\mathbf v$라고 이름 붙일 수 있고, 그 값은 (2,7)이다.
이 두 벡터가 삼각형을 만들어 낸다. 삼각형의 넓이는 이 두 벡터의 외적이 가지는 크기의 반이다.
따라서 넓이는 다음과 같다.
$$S = \frac{1}{2} ||\mathbf a \times \mathbf b || = \frac{6 \cdot 7 - 3 \cdot 2}{2} = 18$$

\begin{figure}[h!]
  \centering
    \includegraphics[width=8cm]{Math_vector/triangleArea.eps}
    \caption{삼각형 세 개의 꼭지점 좌표로 면적 구하기}
    \label{fig:vector:triangleArea}
\end{figure}

\hrule

\vspace{2mm}

예제와 같이 외적은 삼각형의 면적을 구하는 데에 유용하다. 또 다른 응용으로는 평면을 표현하는 것이다.
평면은 그 평면 위의 삼각형으로 표현할 수 있다. 3차원 공간에서 삼각형은 세 개의 점, 즉, 9 개의 원소로 표현된다. 
좀 더 효율적인 방법은 어떠한 것이 있을까? 그것은 평면에 수직인 벡터(이를 법선 벡터라고 한다)로 표현하는 것이다.
하지만, 법선 벡터만 있으면 하나의 평면이 아니라 동일한 방향을 쳐다보는 모든 평면이 된다.
이를 해결하기 위해 법선 벡터와 함께 평면이 지나는 점 하나를 같이 사용하면 유일한 평면이 얻어진다.
그러면 6 개의 실수로 평면을 표현할 수 있다.

이렇게 어떤 평면의 법선 벡터를 구할 때에 가장 빈번히 사용되는 수학적 기법이 ``법선(normal) 벡터"이다.
이 법선 벡터를 구하는 일은 벡터의 외적을 이용하는 것이 가장 일반적인 방법이다. 컴퓨터 그래픽스에서 모델링(modeling)을 위해서
이러한 법선 벡터를 알아야 하는 경우가 빈번하며, 렌더링 과정에서 반사를 계산하는 등의 일에도 표면 법선을 계산해야 하는 일이 빈번하다.
이런 경우 이 장에서 다룬 벡터의 외적이 매우 유용하게 활용된다.




벡터에 대한 깊이 있는 이해를 위해서는 \cite{mortenson1999mathematics,dunn20113d} 등을 참고하라.

\end{document}